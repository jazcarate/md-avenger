\documentclass[a4paper,twocolumn]{article}

\usepackage[T1]{fontenc}
%\usepackage{fourier} % Font de Adobe Utopia
\usepackage[spanish]{babel}
\usepackage{amsmath,amsfonts,amsthm,amssymb}
\usepackage{sectsty}
\usepackage[none]{hyphenat} % No romper palabras

\allsectionsfont{\centering \normalfont\scshape} % Secciones grandes y centradas
\usepackage{hyperref} % Links

%Numeros de tablas, fuguras y secciones como 1.1, 2.4, etc)
\numberwithin{equation}{section}
\numberwithin{figure}{section}
\numberwithin{table}{section}

\setlength\parindent{0pt} % Sin indentacion de parafos

%\newcommand{\refa}[1]{\textsuperscript{\textsf{[\ref{#1}]}}} %Referencias
\newcommand{\refa}[1]{}
\newcommand{\talque}{\;:\,} % En castellano se puede usar ":" o "/"
%----------------------------------------------------------------------------------------
%	Titulo
%----------------------------------------------------------------------------------------

\newcommand{\horrule}[1]{\rule{\linewidth}{#1}} %hrule

\title{	
\normalfont \normalsize 
\textsc{UTN FRBA - Matem\'atica Discreta} \\ [25pt]
\horrule{0.5pt} \\[0.4cm] % Finito arriba
\huge Grafos, D\'igrafos y \'Arboles \\ % TTitulo
\horrule{2pt} \\[0.5cm] % Gordo abajo
}

\author{Alejandra Isola\\
		\small Joaqu\'in Azcarate}
\date{}
\begin{document}

\maketitle % Print the title

%------------------------------------------------------------------------------
%	GRAFOS
%------------------------------------------------------------------------------

\section{Grafos}
Conjunto de puntos o \emph{nodos} unidos por arcos o aristas. Un grafo se describe con una terna $(V,A,\varphi)$ siendo:

\begin{itemize}
	\item $V$: Conjunto de v\'ertices
    \item $A$: Conjunto de aristas
    \item $\varphi$: $A \to \mathcal{V} \subseteq  \big\{ x\in \mathcal{P}(V): |x| \in \{1, 2\}\big\} $
\end{itemize}

\subsection{V\'ertices}
\begin{description}
	\item[V\'ertices adyacentes] $v_i$ es adyacente a $v_j \iff \\ \iff \exists a_k \in A \talque \varphi(a_k)=\{v_i,v_j\}$
    \item[V\'ertices aislados] $v_k$ es aislado $ \iff \\ \iff \nexists a_k \in A \talque v_k \in \varphi(a_k)$
    \item[Istom]\label{istmo} $v$ es istmo $\iff G$ es conexo\refa{conexo} $\Rightarrow \\ \Rightarrow G' = (V - \{v\},A,\varphi/_{V-\{v\}})$ no es conexo\refa{conexo}
\end{description}

\subsection{Aristas}
\begin{description}
	\item[Aristas paralelas]\label{paralela} $a_i$ es paralela a $a_j \iff \\ \iff \varphi(a_i) = \varphi(a_j)$
    \item[Aristas adyacentes]\label{adyacente} $a_i$ es adyacente a $a_j \iff \\ \iff \exists v_k \in V \talque v_k \in \varphi(a_i) \land v_k \in \varphi (a_j)$
    \item[Bucle \emph{o lazo}]\label{bucle} $a_k$ es un bucle $\iff |\varphi(a_k)| = 1$
    \item[Aristas incidentes a un v\'ertice]\label{incidente} $a_i$ y $a_j$ son incidentes en el v\'ertice $v \iff \\ \iff v \in \varphi(a_i) \land v \in \varphi(a_j)$
    \item[Grado o \emph{valencia} de un v\'ertice]\label{grado} $g: V \to \mathbb{Z}$ \\ $g(v_i) = $ cantidad de aristas incidentes\refa{incidente}\\
    	\emph{Nota: Los bucles se cuentan doblemente}
        $$
        	\sum_i g(v_i) = 2 |A|
        $$
	\item[Puente]\label{puente} $a$ es puente $\iff G$ es conexo\refa{conexo} $\Rightarrow \\ \Rightarrow G'=(V,A-\{a\},\varphi/_{A-\{a\}})$ no es conexo\refa{conexo}
    \item[Conjunto desconectante]\label{desconectante} Un conjunto de puentes\refa{puente}
    \item[Conjunto de corte] M\'inimo conjunto desconectante\refa{desconectante}
\end{description}

\subsection{Grafos Particulares}
\begin{description}
	\item[Grafo simple] No tiene aristas paralelas\refa{paralela} ni bucles\refa{bucle}
    \item[Grafo $\mathcal{K}$-regular] $G$ es $\mathcal{K}$-regular $\iff \mathcal{K} \in \mathbb{N}_0 \land \forall v \in V : g(v) = \mathcal{K}$
    \item[Grafo completo] $(K_n) \\ \forall v,w \in V, v \not = w, \exists a \in A \talque \varphi(a) = \{v, w\}$
    \item[Grafo bipartito] $V = V_1 \bigcup V_2, \; V_1 \not = \phi \\ V_2 \not = \phi,\; V1 \bigcap V2 = \phi,\; \forall a_i \in A \talque \\ \varphi(a_i)=\{v_j, v_k\},\; v_j \in V_1 \land v_k \in V2$
    \item[Subgrafos]\label{subgrafo} $G' = (V', A', \varphi') \talque \\ V' \subseteq V, \; A' \subseteq A, \; \varphi': A' \to \mathcal{V'} $
    \item[Componente conexa]\label{componente conexa} Son las clases de equivalencia de la relaci\'on $R: v_i R v_k \iff \\ \iff \exists$ un camino\refa{camino} de $v_i$ a $v_j \lor v_i = v_j$
    \item[Grafo Conexo] \label{conexo}Un grafo con una \'unica componente conexa\refa{componente conexa}
    \item[Conectividad] El menor n\'umero de istmos\refa{istmo}
    \item[Grafos planos] Grafo que se puede dibujar sin que se crucen aristas
\end{description}

\subsection{Caminos y Ciclos}
\begin{description}
	\item[Camino]\label{camino} Sucesi\'on de aristas adyacentes\refa{adyacente} distintas
    \item[Ciclo \emph{o circuito}]\label{ciclo} Camino\refa{camino} cuyo v\'ertice inicial conicide con el final
    \item[Longitud] Cantidad de aristas que componen un camino\refa{camino}
    \item[Camino simple]\label{camino simple} Camino\refa{camino} con todas sus aristas distintas
   \item[Camino elemental] Camino\refa{camino} con todos sus v\'ertices distintos
   \item[Camino euleriano] Camino\refa{camino} que pasa por \emph{todas} las aristas \emph{una sola vez}
    	\\ CNyS:
        	\begin{itemize}
            	\item Ser conexo\refa{conexo}
                \item $\forall v_i : g(v_i) = 2k; k \in \mathbb{N}$ o a lo sumo dos v\'ertices de grado\refa{grado} impar
            \end{itemize}
   \item[Ciclo euleriano] Ciclo\refa{ciclo} que pasa por \emph{todas} las aristas \emph{una sola vez}
        	\\ CNyS:
        	\begin{itemize}
            	\item Ser Conexo\refa{conexo}
                \item $\forall v_i : g(v_i) = 2k$
            \end{itemize}
	\item[Camino/Ciclo hamiltoniano] Camino\refa{camino}/Ciclo\refa{ciclo} que pasa por \emph{todos} los v\'ertices \emph{una sola vez}   
\end{description}

\subsection{Representaci\'on Matricial}
\begin{description}
	\item[Matriz de incidencia] $Mi^{|V|x|A|} \\ a_{ij} = \left\{
\begin{array}{c l}
 1 & v_i\textup{ es incidente\refa{incidente} a } a_j\\
 0 & v_i\textup{ no es incidente\refa{incidente} a } a_j\\
\end{array}
\right.
$
    \item[Matriz de adyacencia]$Ma^{|V|x|V|} \\ a_{ij} = \left\{
\begin{array}{c l}
 1 & v_i\textup{ es adyacente\refa{adyacente} a } v_j\\
 0 & v_i\textup{ no es adyacente\refa{adyacente} a } v_j\\
\end{array}
\right.
$
\end{description}

\subsection{Isomorfismo}\label{isomorfo}
Dado $G_1 = (V_1, A_1, \varphi_1)$ y $G_2 = (V_2, A_2, \varphi_2)$ \\
$G_1$ es isomorfo a $G_2 \iff \\ \iff \exists f:V_1 \to V_2 \,\land \,h : A_1 \to A_2$ , ambas funciones biyectivas y $\forall a \in A_1 \talque \varphi_2\big{(} h(a) \big{)} = f\big{(} \varphi_1(a) \big{)}$

\subsection{Teorema de Kuratowski}
Un grafo es plano $\iff$ no contiene ning\'un subgrafo\refa{subgrafo} isomorfo\refa{isomorfo} al $K_{3,3}$ o al $K_5$

%------------------------------------------------------------------------------
%	DIGRAFOS
%------------------------------------------------------------------------------

\section{D\'igrafos}
Conjunto de puntos o \emph{nodos} unidos por arcos o aristas \emph{direccionadas}\\
Un grafo se describe con una terna $(V,A,\delta )$, siendo:

\begin{itemize}
	\item $V$: Conjunto de v\'ertices
    \item $A$: Conjunto de aristas
    \item $\delta $: $A \to VxV$
\end{itemize}

\subsection{Aristas}
\begin{description}
	\item[Aristas paralelas] $a_i$ es paralela a $a_j \iff \\ \iff \delta(a_i) = \delta(a_j)$
    \item[Aristas antiparalelas] $a_i$ es antiparalela a $a_j \\ \iff \delta(a_i) = (v_1, v_2) \Rightarrow  \delta(a_j) = (v_2, v_1)$
    \item[Bucle \emph{o lazo}] $a_k$ es un bucle $\iff \delta(a_k) = (v,v)$
	\item[Grado positivo ($g^+$)] Cantidad de aristas que ``\emph{entran}'' al v\'ertice
    \item[Grado negativo ($g^-$)] Cantidad de aristas que ``\emph{salen}'' del v\'ertice
    \item[Grado total ($g$)] $g^+ + g^-$
    \item[Grado neto ($g_N$)] $g^+ - g^-$
    \item[Pozo] $g^-(v) = 0$
    \item[Fuente] $g^+(v) = 0$
\end{description}

\subsection{Grafo Asociado}
Dado el d\'igrafo $(V,A,\delta)$, se define el grafo asociado $(V,A,\varphi)\talque$
$$
	\forall a_i \in A,\; \varphi(a_i) = \{Primero(\delta(a_i)), Segundo(\delta(a_i))\}
$$

\subsection{D\'igrafos Particulares}
\begin{description}
	\item[D\'igrafos conexo] Aquel cuyo grafo asociado sea conexo\refa{conexo}
    \item[D\'igrafos fuertemente conexo] $\forall$ par de v\'ertices, $\exists$ un camino\refa{camino} entre ellos
\end{description}

\subsection{Caminos y Ciclos}
Se debe respetar el sentido de las aristas
\begin{description}
	\item[Ciclo euleriano] CNyS: $\forall v \in V \;: g^+(v) = g^-(v)$
\end{description}

\subsection{Representaci\'on Matricial}
\begin{description}
	\item[Matriz de incidencia] $Mi^{|V|x|A|} \\ a_{ij} = \left\{
\begin{array}{c l}
 1 & \exists v_x \in V \talque \delta(a_j) = (v_i, v_x)\\
 -1 & \exists v_x \in V \talque \delta(a_j) = (v_x, v_i)\\
 0 & \textup{si $v_i$ no es extremo de } a_j\\
\end{array}
\right.
$
    \item[Matriz de adyacencia]$Ma^{|V|x|V|} \\ a_{ij}= \left\{
\begin{array}{c l}
 1 & \exists a \in A \talque \delta(a) = (v_i, v_j)\\
 0 & \nexists a \in A \talque \delta(a) = (v_i, v_j)\\
\end{array}
\right.
$
\end{description}

%------------------------------------------------------------------------------
%	ARBOLES
%------------------------------------------------------------------------------

\section{\'Arboles}
Grafo conexo y sin ciclos\\
\emph{Un grafo donde existe un \'unico camino simple entre todo par de v\'ertices}
\begin{itemize}
	\item Todas las aristas son puente\refa{puente}
    \item $|V| = |A| +1$
\end{itemize}
\begin{description}
	\item[Hoja] $g(v) = 1$
\end{description}
\subsection{\'Arbol Dirigido}
\begin{description}
    \item[Antecesor] $v$ es antecesor de $w \iff \\ \iff \exists ! $ camino simple\refa{camino simple} de $v$ a $w$
    \item[Sucesor] $v$ es sucesor de $w \iff \\ \iff w$ es antecesor de $v$
    \item[Padre] $v$ es padre de $w \iff \exists a_k \talque \delta(a_k) = (v, w)$
    \item[Hijo] $v$ es hijo de $w \iff w$ es padre de $v$
    \item[Hermanos] $v$ es hermano de $w \iff v$ y $w$ tienen el mismo padre
    \item[Nivel] Cantidad de padres tiene hasta llegar a la raiz
    \item[Altura] M\'aximo nivel
\end{description}

\subsection{\'Arboles Particulares}
\begin{description}
    \item[$\mathcal{N}$-ario] $\forall v \in V \talque g^-(v) \leq \mathcal{N}$
    \item[$\mathcal{N}$-ario regular] $\forall v \in V \talque g(v)=0 \lor g(v)=\mathcal{N}$
    \item[$\mathcal{N}$-ario regular pleno \emph{o completo}] $\mathcal{N}$=ario regular donde  todas las hojas se hallan en el mismo nivel
\end{description}

\subsection{Recorrido de \'Arboles}
\begin{description}
    \item[Orden previo \emph{o pre-orden}] Notaci\'on polaca
		\begin{enumerate}
      		\item Nomrar la raiz
      		\item Recorre el sub-\'arbol izquierdo
            \item Recorre el sub-\'arbol derecho
    	\end{enumerate}
    \item[Orden simetrico \emph{o in-orden}] Notaci\'on usual o infija
		\begin{enumerate}
      		\item Recorre el sub-\'arbol izquierdo
            \item Nomrar la raiz
            \item Recorre el sub-\'arbol derecho
    	\end{enumerate}
    \item[Orden posterior \emph{o post-orden}] Notaci\'on polaca inversa
		\begin{enumerate}
      		\item Recorre el sub-\'arbol izquierdo
            \item Recorre el sub-\'arbol derecho
            \item Nomrar la raiz
    	\end{enumerate}
\end{description}

\end{document}