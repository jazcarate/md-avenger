\documentclass[a4paper]{article}

\usepackage[T1]{fontenc}
%\usepackage{fourier} % Font de Adobe Utopia
\usepackage[spanish]{babel}
\usepackage{amsmath,amsfonts,amsthm}

\usepackage{sectsty}
\allsectionsfont{\centering \normalfont\scshape} % Secciones grandes y centradas
\usepackage{hyperref} % Links

%Numeros de tablas, fuguras y secciones como 1.1, 2.4, etc)
\numberwithin{equation}{section}
\numberwithin{figure}{section}
\numberwithin{table}{section}

\setlength\parindent{0pt} % Sin indentacion de parafos

%\newcommand{\refa}[1]{\textsuperscript{\textsf{[\ref{#1}]}}} %Referencias
\newcommand{\refa}[1]{}

%----------------------------------------------------------------------------------------
%	Titulo
%----------------------------------------------------------------------------------------

\newcommand{\horrule}[1]{\rule{\linewidth}{#1}} %hrule

\title{	
\normalfont \normalsize 
\textsc{UTN FRBA - Matematica Discreta} \\ [25pt]
\horrule{0.5pt} \\[0.4cm] % Finito arriba
\huge Grafos, D\'igrafos y \'Arboles \\ % TTitulo
\horrule{2pt} \\[0.5cm] % Gordo abajo
}

\author{Alejandra Isola\\
		\small Joaqu\'in Azcarate}
\date{}
\begin{document}

\maketitle % Print the title

%------------------------------------------------------------------------------
%	GRAFOS
%------------------------------------------------------------------------------

\section{Grafos}
Conjunto de puntos o \emph{nodos} unidos por arcos o aristas.\\
Un grafo se describe con una terna $(V,A,\varphi)$, siendo:

\begin{itemize}
	\item $V$: Conjunto de v\'ertices
    \item $A$: Conjunto de aristas
    \item $\varphi$: $A \to \mathcal{V} \subseteq  \{x\in \mathcal{P}(V): |x|=2\} $
\end{itemize}

\subsection{V\'ertices}
\begin{description}
	\item[V\'ertices Adyacentes] $v_i$ es adyacente a $v_j \iff \exists a_k \in A / \varphi(a_k)=\{v_i,v_j\}$
    \item[V\'ertices Aislados] $v_k$ es aislado $\iff \not \exists a_k \in A / v_k \in \varphi(a_k)$
    \item[Istom]\label{istmo} $v$ es istmo $\iff G$ es conexo\refa{conexo} $\Rightarrow G' = (V - \{v\},A,\varphi)$ no es conexo\refa{conexo}
\end{description}

\subsection{Aristas}
\begin{description}
	\item[Aristas Paralelas]\label{paralela} $a_i$ es paralela a $a_j \iff \varphi(a_i) = \varphi(a_J)$
    \item[Aristas Adyacentes]\label{adyacente} $a_i$ es adyacente a $a_j \iff \exists v_k \in V / v_k \in \varphi(a_i) \land v_k \in \varphi (a_j)$
    \item[Bucle \emph{o lazo}]\label{bucle} $a_k$ es un bucle $\iff \varphi(a_k) = \{v,v\}$
    \item[Aristas incidentes a un v\'ertice]\label{incidente} $a_i$ y $a_j$ son incidentes en el v\'ertice $v \iff \\ \iff v \in \varphi(a_i) \land v \in \varphi(a_j)$
    \item[Grado o \emph{valencia} de un v\'ertice]\label{grado} $g: V \to \mathbb{Z} / g(v_i) = $ cantidad de aristas incidentes\refa{incidente}\\
    	\emph{Nota: Los bucles se cuentan doblemente}
        $$
        	\sum_i g(v_i) = 2 |A|
        $$
	\item[Puente]\label{puente} $a$ es puente $\iff G$ es conexo\refa{conexo} $\Rightarrow G'=(V,A-\{a\},\varphi/_{A-\{a\}) }$ no es conexo\refa{conexo}
    \item[Conjunto desconectante]\label{desconectante} Un conjunto de puentes\refa{puente}
    \item[Conjunto de Corte] M\'inimo conjunto desconectante\refa{desconectante}
\end{description}

\subsection{Grafos Particulares}
\begin{description}
	\item[Grafo Simple] No tiene aristas paralelas\refa{paralela} ni bucles\refa{bucle}
    \item[Grafo $K$-Regular] $G$ es $K$-Regular $\iff K \in \mathbb{N}_0 \land \forall v \in V : g(v) = k$
    \item[Grafo Completo] $(K_n) \; \forall v,w \in V : v \not = w \Rightarrow \exists a \in A : \varphi(a) = \{v, w\}$
    \item[Grafo Bipartito] $V = V_1 \bigcup V_2; \; V_1 \not = \phi \land V_2 \not = \phi \land V_1 \bigcup V_2 \not = \phi \land \forall a_i \in A: \varphi(a_i)=\{v_j, v_k\}/ \\ v_j \in V_1 \land v_k \in V2$
    \item[Subgrafos]\label{subgrafo} $G' = (V', A', \varphi/_{A'}) \;/\;V' \subseteq V, A' \subseteq A$
    \item[Componente Conexa]\label{componente conexa} Son las clases de equivalencia de la relaci\'on $R: v_i R v_k \iff \exists$ un camino\refa{camino} de $v_i$ a $v_j \lor v_i = v_j$
    \item[Grafo Conexo] \label{conexo}Un grafo con una \'unica componente conexa\refa{componente conexa}
    \item[Conectividad] El menor n\'umero de istmos\refa{istmo}
    \item[Grafos Planos] Grafo que se puede dibujar sin que se crucen aristas
\end{description}

\subsection{Caminos y Cliclos}
\begin{description}
	\item[Camino]\label{camino} Sucesi\'on de aristas adyacentes\refa{adyacente} distintas
    \item[Ciclo \emph{o circuito}]\label{ciclo} Camino\refa{camino} cuyo v\'ertice inicial conicide con el final
    \item[Longitud] Cantidad de aristas que componen un camino\refa{camino}
    \item[Camino Simple]\label{camino simple} Camino\refa{camino} con todos los v\'ertices distintos
    \item[Camino/Ciclo hamiltoniano] Camino\refa{camino}/Ciclo\refa{ciclo} que pasa por \emph{todos} los v\'ertices \emph{una sola vez}
    \item[Camino Euleriano] Camino\refa{camino} que pasa por \emph{todas} las aristas \emph{una sola vez}
    	\\ CNyS:
        	\begin{itemize}
            	\item Ser Conexo\refa{conexo}
                \item $\forall v_i : g(v_i) = 2k; k \in \mathbb{n}$ o a lo sumo dos v\'ertices de grado\refa{grado} impar
            \end{itemize}
    \item[Ciclo euleriano] Ciclo\refa{ciclo} que pasa por \emph{todas} las aristas \emph{una sola vez}
        	\\ CNyS:
        	\begin{itemize}
            	\item Ser Conexo\refa{conexo}
                \item $\forall v_i : g(v_i) = 2k$
            \end{itemize}
\end{description}

\subsection{Representaci\'on Matricial}
\begin{description}
	\item[Matriz de incidencia] $Ma_{ij}^{|V|x|A|} = \left\{
\begin{array}{c l}
 1 & v_i\textup{ es incidente\refa{incidente} a } a_j\\
 0 & v_i\textup{ no es incidente\refa{incidente} a } a_j\\
\end{array}
\right.
$
    \item[Matriz de adyacencia]$Ma_{ij}^{|V|x|V|} = \left\{
\begin{array}{c l}
 1 & v_i\textup{ es adyacente\refa{adyacente} a } v_j\\
 0 & v_i\textup{ no es adyacente\refa{adyacente} a } v_j\\
\end{array}
\right.
$
\end{description}

\subsection{Isomorfismo}\label{isomorfo}
Dado $G_1 = (V_1, A_1, \varphi_1)$ y $G_2 = (V_2, A_2, \varphi_2)$ \\
$G_1$ es isomorfo a $G_1 \iff \exists f:V_1 \to V_2 \land h: A_1 \to A_2$ ambas funciones biyectivas y $\forall a \in A_1 :\; \varphi_2( \;h(a) \;) = f(\; \varphi(a) \;)$

\subsection{Teorema de Kuratowski}
Un grafo es plano $\iff$ no contiene ning\'un subgrafo\refa{subgrafo} isomorfo\refa{isomorfo} al $K_{3,3}$ o al $K_5$

%------------------------------------------------------------------------------
%	DIGRAFOS
%------------------------------------------------------------------------------

\section{D\'igrafos}
Conjunto de puntos o \emph{nodos} unidos por arcos o aristas \emph{direccionadas}\\
Un grafo se describe con una terna $(V,A,\delta )$, siendo:

\begin{itemize}
	\item $V$: Conjunto de v\'ertices
    \item $A$: Conjunto de aristas
    \item $\delta $: $A \to VxV$
\end{itemize}

\subsection{Grado}
\begin{description}
	\item[Grado Positivo ($g^+$)] Cantidad de aristas que ``entran'' al v\'ertice
    \item[Grado Negativo ($g^-$)] Cantidad de aristas que ``salen'' del v\'ertice
    \item[Grado Total ($g$)] $g^+ + g^-$
    \item[Grado Neto ($g_N$)] $g^+ - g^-$
    \item[Pozo] $g^-(v) = 0$
    \item[Fuente] $g^+(v) = 0$
\end{description}

\subsection{Grafo asociado}
Dado el D\'igrafo $(V,A,\delta)$, se define el Grafo asociado $(V,A,\phi)$, tal que:
$$
	\forall a_i \in A,\; \phi(a_i) = \{Primero(\delta(a_i)), Segundo(\delta(a_i))\}
$$

\subsection{D\'igrafos}
\begin{description}
	\item[D\'igrafos Conexo] Aquel cuyo grafo asociado sea conexo\refa{conexo}
    \item[D\'igrafos Fuertemente Conexo] $\exists$ camino\refa{camino} entre todo par de v\'ertices
\end{description}

\subsection{Caminos y Ciclos}
Se debe respetar el sentido de las aristas
\begin{description}
	\item[Ciclo de Euler] CNyS: $\forall v \in V \;: g^+(v) = g^-(v)$
\end{description}

\subsection{Representaci\'on Matricial}
\begin{description}
	\item[Matriz de incidencia] $Ma_{ij}^{|V|x|A|} = \left\{
\begin{array}{c l}
 1 & \exists v_x \in V / \delta(a_j) = (v_i, v_x)\\
 -1 & \exists v_x \in V / \delta(a_j) = (v_x, v_i)\\
 0 & \textup{si $v_i$ no es extremo de } a_j\\
\end{array}
\right.
$
    \item[Matriz de adyacencia]$Ma_{ij}^{|V|x|V|} = \left\{
\begin{array}{c l}
 1 & \exists a \in A \;: \delta(a) = (v_i, v_j)\\
 0 & \not \exists a \in A \;: \delta(a) = (v_i, v_j)\\
\end{array}
\right.
$
\end{description}

%------------------------------------------------------------------------------
%	ARBOLES
%------------------------------------------------------------------------------

\section{\'Arboles}
Grafo conexo y sin ciclos\\
\emph{Un grafo donde existe un \'unico camino simple entre todo par de v\'ertices}
\begin{itemize}
	\item Todas las aristas son puente\refa{puente}
    \item $|V| = |A| +1$
\end{itemize}
\begin{description}
	\item[Hoja] $g(v) = 1$
    \item[Antecesor] $v$ es antecesor de $w \iff \exists ! $ camino simple\refa{camino simple} de $v$ a $w$
    \item[Sucesor] $v$ es sucesor de $w \iff w$ es antecesor de $v$
    \item[Padre] $v$ es padre de $w \iff \exists a_k \;/ \varphi(a_k) = \{v, w\}$
    \item[Hijo] $v$ es hijo de $w \iff w$ es padre de $v$
    \item[Hermanos] $v$ es hermano de $w \iff v \land w$ tienen el mismo padre
    \item[Nivel] Cuantos padres tiene hasta llegar a la raiz
    \item[Altura] M\'aximo nivel
\end{description}


\end{document}